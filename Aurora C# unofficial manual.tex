%!TeX spellcheck = en_US
\documentclass[10pt,a4paper,oneside]{article}
\usepackage[utf8]{inputenc}
\usepackage[T1]{fontenc}
\usepackage[english]{babel}
\usepackage{mathtools}
\usepackage{indentfirst}
\usepackage{amsfonts}
\usepackage{amssymb}
\usepackage{graphicx}
\usepackage[left=2.00cm, right=2.00cm, top=2.00cm, bottom=2.00cm]{geometry}
\usepackage{hyperref}
\hypersetup{
	colorlinks,
	citecolor=black,
	filecolor=black,
	linkcolor=black,
	urlcolor=blue
}



\author{Written by Steve Walmsley and organized by Aurora C\# forum community}
\title{Aurora C\# unofficial manual}


\begin{document}
\maketitle
\newpage
\tableofcontents

\newpage
\part{Introduction}
\section{What is Aurora C\#}
Aurora C\# is turn based 4X\footnote{4X - eXplore, eXpand, eXploit, eXterminate}
computer game made solely by one man: Steve Walmsley as a hobby project in spare time.
It's modern version of it's predecessor „Aurora 4X” called sometimes „Aurora VB6” from
Visual Basic 6 - language it was written in. Aurora C\# is writen in... C\# more modern
programming language which resulted in dramatic increse of performance.

Aurora C\# is free game distributed thru forum under address
\href{http://aurora2.pentarch.org/}{http://aurora2.pentarch.org/} where you can find
more information about game and ask qestions about mechanics to more experienced
players and the developer himself.


\section{About this document}
This document wasn't be possible without Steve Walmsley who wrote devlogs during game
developement and forum community, especially user named Demonides who gave idea and
started consolidation of information scattered around forum.

This document is based on his and othes work. Thank you! You can find original topic
here: \href{http://aurora2.pentarch.org/index.php?topic=10666.0}{C\# Aurora Changes List v1.12 / Table of Contents}

This document and it's source in LaTeX is distributed under \href{https://creativecommons.org/licenses/by-nc/4.0/}{Creative Commons Attribution-NonCommercial 4.0 International Public License}. You can help improve it
on \href{https://github.com/Pawel82S/Aurora-C-Sharp-manual}{GitHub} or Aurora official \href{http://aurora2.pentarch.org/}{forum}. Newest versions of this document in PDF format are availble \href{https://drive.google.com/drive/u/1/folders/1_AWEpFhS7e7ouguzrx8u0gnBHPhmjHoI}{here}

\newpage
\part{Diplomacy}
\section{Basic Framework}\label{1_basic_framework}
Original post can be found
\href{http://aurora2.pentarch.org/index.php?topic=8495.msg118258#msg118258}{here}.

\subsection{Diplomacy module}
The Diplomacy Module is new for C\# Aurora and replaces Diplomatic Teams. It also affects communication attempts. The module is 30 HS, costs 300 BP and requires 50 crew. The minerals required are Corbomite, Mercassium and Vendarite. It is a starting technology in both TN and conventional starts.

\subsection{Communication}
For communication checks to take place both sides must have ships and/or populations in the same system and both sides must be able to detect the other. Communication checks will only take place if both sides have a status of „Attempting Communication”. In other words, you can't translate their language if they refuse to talk to you. Diplomacy cannot take place until full communication is established. Alien races may take exception to your presence in this situation, based on a number of factors will be covered in a future post.

For communication attempts, the highest Communication bonus of any commander of a ship with a Diplomacy module in one or more of the contact systems will boost any positive results achieved through the communication process (which is otherwise the same as VB6). If no Diplomacy module is present in any of the contact systems or the commander has no communication bonus, any positive gains toward full communication are halved.

\subsection{Basic diplomacy}
Basic diplomacy follows similar principles to VB6 Aurora. Actions by each side generate positive or negative diplomatic points. As the total of diplomatic points goes above or below certain thresholds, high level treaties (trade, sharing of data, etc.) are put in place and the general level of cooperation changes (hostile, neutral, friendly, allied).

The primary method of generating diplomatic points is via the Diplomacy module. The module must be located in a system where the target alien race has ships and/or populations and both sides must be able to detect the other. Diplomacy can only take place when full communication has been established. The highest Diplomacy bonus of any commander of a qualifying ship is used. The number of points generated per year is as follows:

\[ Diplomacy Points = (Diplomacy Bonus * 4 + 1) * 100 * (1 - \frac{Target Racial Xenophobia}{100}) \]

For example, an officer with 20\% Diplomacy trying to influence an alien race with Xenophobia of 40 would have the following calculation: \( (0.2 * 4 + 1) * 100 * 0.6 = 108 \) Points.

If there is contact but no Diplomacy module in a contact system or the commander has no Diplomacy bonus, then no points are generated from this process (although other factors may generate points - covered in a future post).

If there is no contact at all, even via civilian ships, then Diplomacy Points will move toward zero, from either direction. The annual rate of change is the Xenophobia of the viewing race when the starting point is positive and 100 – Xenophobia when the starting point is negative. For example, the view of a race with 25 Xenophobia will only fall 25 points when the starting point is positive but will rise by 75 points when the starting point is negative. Low Xenophobia races are quicker to forgive transgressions and vice versa.

Existing treaties or diplomatic statuses will improve relationships over time. Different treaties have a base influence that is measured in diplomatic points per year multiplied by \( 1 - \frac{Racial Xenophobia}{100} \). For example, a trade treaty has a base influence of 100 diplomatic points per year. If two races have respective Racial Xenophobia of 30 and 60, then while a treaty is in place the view of the first race will improve by 70 diplomatic points per year while the view of the second ace will improve by 40. It takes longer to build trust with higher Xenophobia races.

Trade, Geological and Gravitational treaties all have a base influence of 100. A research treaty has a base influence of 200. A diplomatic status of friendly has a base influence of 100, while a diplomatic status of allied has a base influence of 200.

Positive and Negative diplomatic points will be gained through other events, many of which will be defined in future posts. An example of a negative impact is combat. Negative diplomatic points are suffered due to damage inflicted by an alien race using the following rules:\newline
Each point of damage from a hit that only damages shields: 0.1\newline
Each point of damage from a hit that causes armour damage but not internal: 0.25\newline
Each point of damage from a hit that causes internal damage: 1.0\newline
Each point of space-based damage to populations, ground forces or shipyards: 1.0\newline
Each ton of ground forces destroyed in ground-based combat: 0.01\newline

If diplomatic relations are above the hostile level (-100), then even a single point of damage through combat will reduce relations to that point. However a period of mutual non-interaction following a small clash will probably return the diplomatic status to neutral. For example, if communications are established you may ask a survey ship to leave your system (mechanics in a future post). If that didn't work or you did not have communication, you can slightly damage that ship. An unarmed ship would retreat from hostile aliens and the immediate impact would be the alien race treating you as hostile. However, with no further combat in the short term, the status would soon return to a wary neutrality. Future communication and diplomacy would still be an option. Larger wars are harder to resolve but peace treaties will be covered in a future post.

\section{Intrusion into NPR territory}\label{2_intrusion_into_npr}
Original post can be found
\href{http://aurora2.pentarch.org/index.php?topic=8495.msg118318#msg118318}{here}.
\newline\newline
In each construction phase, each NPR will determine a value for each known system. In order of ascending importance, the values are: Alien Controlled, Neutral, Claimed, Secondary, Primary, Core, Capital. The value is calculated on a number of different factors, including existing population and installations, whether it is a logistics node, mining potential, terraforming potential and proximity to other important systems. Neutral is the default state for a system in which the NPR has no current interest, while Alien Controlled is a system which the NPR acknowledges is in the territory of another race as a result of accepting a claim from that race (see section \ref{3_claiming_npr_systems}).

If you have forces or a population in a system that has at least Secondary value to an NPR, you are detected and you are currently viewed as neutral or friendly, the NPR will issue a warning which will appear as an event. This will still happen even if you haven't detected any NPR forces. You will be notified which fleet or population received the message. If communication has not been established, you will receive notification of an „unintelligible communication of unknown origin”. If you have established communication, the text will reflect the severity of the situation.

This communication can be as mild as a suggestion that your forces leave in the near future and as strong as demanding you depart immediately or be fired upon. There are five levels of severity for messages and the one chosen by the NPR primarily depends on the 'Threat Level' (see below), although it may also issue a stronger warning at lower threat levels if the NPR believes that war will soon follow without a player withdrawal.

The threat level is based on three factors; the NPR’s estimate of the value of the system, any status modifiers due to the existing diplomatic relations and the Xenophobia of the NPR. This is calculated as follows:
\[ Threat Level = Base Threat Level * Status Modifier * \frac{Racial Xenophobia}{100}) \]
\begin{center}
	\begin{tabular}{|l|r|}
		\hline
		\multicolumn{2}{|c|}{\textbf{Base Threat Level}} \\
		\hline
		Secondary & 2.5 \\
		\hline
		Primary & 5 \\
		\hline
		Core & 10 \\
		\hline
		Capital & 20 \\
		\hline
	\end{tabular}
\end{center}
\textbf{Status Modifiers:}\newline
Friendly Status = 0.5\newline
Neutral with Diplomatic Points >= 1\newline
Neutral with Diplomatic Points < 0 = 2\newline

In addition to the messages, the threat levels generate a negative impact on diplomatic relations. The penalty in diplomatic points for intrusion into NPR territory is based on the Threat Level above plus the ships and population that the NPR can detect. The calculation for the annual point penalty is as follows:
\[ DPP\footnote{DPP - Diplomatic Point Penalty} = \sqrt{Total Detected Ship Tonnage + Total Detected Population EM Signature * 10 * Threat Level} \]

Each construction phase, the diplomatic penalty applied is equal to the annual penalty multiplied by \( \frac{Construction Phase Length}{Year} \)

Shipping Line vessels will be ignored for this purpose if a trade treaty is in force. NPRs will treat ships without military engines that have not demonstrated any weapon capability as 10\% of their normal tonnage. If at least one ship is detected, the minimum rating for Detected Ship Tonnage will be 1000 tons. If at least one population is detected, the minimum rating for Population EM Signature will be 100. NPRs deduct 10,000 tons from the tonnage of one Diplomatic Ship (see section \ref{8_diplomatic_ships}) per system for threat purposes if that class type has never fired weapons and the Diplomatic Ship is in a non-Core system. If the NPR only has one system, it is not treated as core for this purpose.

This table shows the diplomatic point penalties for different ship tonnages in different value systems, assuming an NPR Xenophobia of 50. For populations, use EM Signature * 10 for ‘Tonnage’.
\begin{center}
	\begin{tabular}{|c|c|c|c|c|c|c|c|c|c|}
		\hline
		\multicolumn{2}{|c}{}  & \multicolumn{4}{|c|}{\textbf{Annual Diplomacy Penalty}} & \multicolumn{4}{c|}{\textbf{Construction Phase Penalty}}  \\
		\hline
		\textbf{Tonnage} & \( \sqrt{\textbf{Tonnage}} \) & \textbf{Secondary} & \textbf{Primary} & \textbf{Core} & \textbf{Capital} & \textbf{Secondary} & \textbf{Primary} & \textbf{Core} & \textbf{Capital} \\
		\hline
		1000 & 31.6 & 39.5 & 79.1 & 158.1 & 316.2 & 0.5 & 1.1 & 2.2 & 4.3 \\
		\hline
		3000 & 54.8 & 68.5 & 136.9 & 273.9 & 547.7 & 0.9 & 1.9 & 3.8 & 7.5 \\
		\hline
		10000 & 100.0 & 125.0 & 250.0 & 500.0 & 1000.0 & 1.7 & 3.4 & 6.8 & 13.7 \\
		\hline
		30000 & 173.2 & 216.5 & 433.0 & 866.0 & 1732.1 & 3.0 & 5.9 & 11.9 & 23.7 \\
		\hline
		100000 & 316.2 & 395.3 & 790.6 & 1581.1 & 3162.3 & 5.4 & 10.8 & 21.7 & 43.3 \\
		\hline
		300000 & 547.7 & 684.7 & 1369.3 & 2738.6 & 5477.2 & 9.4 & 18.8 & 37.5 & 75.0 \\
		\hline
		1000000 & 1000.0 & 1250.0 & 2500.0 & 5000.0 & 10000.0 & 17.1 & 34.2 & 68.5 & 137.0 \\
		\hline
	\end{tabular}
\end{center}

The warning message is issued during the first construction phase after detection and repeated during each subsequent construction phase where the violation still exists. Allied Races do not receive warnings as they can freely enter the NPR territory. Hostile races do not receive warnings as they are attacked instead. Trading will allow some exceptions to the rules above and I'll cover that in a future post. I will also cover situations where the NPR considers claiming a system with a large existing player population in the 'Alien Controlled' update.

\section{Claiming systems from NPRs}\label{3_claiming_npr_systems}
Original post can be found
\href{http://aurora2.pentarch.org/index.php?topic=8495.msg118362#msg118362}{here}.
\newline\newline
In the same way that NPRs can warn players to leave a system, a player can warn an NPR. On the Intelligence and Foreign Relations window, there is a tab for Known Systems for each NPR. You can select a system and then set a 'Protection Status' for the selected system in connection with the selected NPR. Alternatively, you can set a protection status for the system on the galactic map and that status will be set for any alien race when they are first detected in the system. The six statuses are shown below with their ‘Demand Number’ (0-5) and their ‘Demand Strength’.

\begin{center}
	\begin{tabular}{|l|l|}
		\hline
		\multicolumn{2}{|c|}{\textbf{Protection Status}} \\
		\hline
		0 & No Protection: 0 \\
		\hline
		1 & Suggest Leave: 1 \\
		\hline
		2 & Request Leave: 1.41 \\
		\hline
		3 & Request Leave Urgently: 1.73 \\
		\hline
		4 & Demand Leave: 2 \\
		\hline
		5 & Demand Leave with Threat: 2.24 \\
		\hline
\end{tabular}
\end{center}


If you set a status for a specific combination of system and NPR, then if that NPR is detected by you in that system during a construction phase it will be informed of your demand unless it is already allied or hostile.

The impact of the message on the NPR decision to accept or reject your demand is shown by the ‘Demand Strength’ in the list above, which is the square root of the ‘Demand Number’. A Request is 41\% more likely to work than a Suggestion, while a Demand with Threat is 2.24x more effective than a Suggestion. The Demand Value represents the idea that, from the perspective of the NPR, the forcefulness of your language may represent a willingness to use force.

While the strength of your demand plays a part in the NPR decision, it also has a significant effect on relations with that NPR. So a higher demand might increase the chance the NPR will leave, but it also increases the chance of starting a war. If you demand the NPR leaves a system it doesn't care about you will cause fairly minor damage, but you could have made a polite request and it may have left with hardly any impact on relations. If you demand an NPR abandons what it regards as a primary system, that might work if you have a significant military advantage and the NPR is aware of it, but it might also cause the NPR to open fire immediately.

\begin{center}
	\begin{tabular}{|r|l|}
		\hline
		\multicolumn{2}{|c|}{\textbf{System Values}} \\
		\hline
		Neutral & 1 \\
		\hline
		Claimed & 2 \\
		\hline
		Secondary & 3 \\
		\hline
		Primary& 4 \\
		\hline
		Core & 5 \\
		\hline
		Capital & 6 \\
		\hline
	\end{tabular}
\end{center}

This relationship impact is equal to \( Demand Number^{2} * System Value^{2} * \frac{Xenophobia}{50} \). So a demand to leave a primary system would have the impact of \( 256 * \frac{Xenophobia}{50} \), while a request to leave a claimed system would have an impact of only \( 16 * \frac{Xenophobia}{50} \).

The demand will be rejected if the NPR has not detected populations of your race with a total EM signature of \( 10 * Xenophobia \) or more. The NPR will base this on actual populations, not currently detected populations, as it is assumed you will provide the necessary evidence to back up your demand.

Otherwise, the demand will be accepted based on Demand Value plus the following additional factors:

\textbf{Accessible System Value}\newline
For each system that would no longer be accessible if the claim was accepted, including the target system, a value is assigned equal to \( \frac{System Value^{2}}{4} \). For example, a Claimed system is worth 1, a Secondary system is worth 2.25, a Primary system is worth 4, etc. Each individual system value is calculated first and then the results are summed.

\textbf{Military Advantage}\newline
This assessment depends on the total size of your military forces that have been detected by the NPR during the last five years in comparison to its own (with an assumption of some as-yet-unseen forces) and its assessment of relative technology based on its observation of your ships. I don't want to go into too much detail on the Player Military Advantage, but if the NPR believes the racial balance of forces is equal, Player Military Advantage will be equal to 1. For the NPR to believe you have an advantage, it will need to see some firepower. This is based on total known forces, not local known forces, so generating a high Military Advantage number is difficult unless you show off a large portion of your forces. You won't be able to simply send in a survey ship and ask the NPR to move out

\textbf{Population Factor}\newline
This is equal to:
\[ \sqrt{\frac{Total EM Signature of Player Populations in System}{Total EM Signature of NPR Populations in System}} \]
However, this factor can never be higher than the fourth root of \( \frac{Total EM Signature of Player Populations in System}{100} \). For example, if the player had 1000 EM Signature and the NPR has 200 EM Signature, the factor would be 1.78 (because the fourth root of (1000/100) is lower than \( \sqrt{1000 / 200} \). This is to limit the advantage when the populations are relatively small or the NPR has no populations. Population Factor is the best ’peaceful option’ as demonstrating a large population is much more likely to achieve a decision in your favour.

\textbf{Resistance}\newline
\[ \frac{(Xenophobia + Militancy + Determination)}{150} \]
If the NPR has low militancy, low determination and low xenophobia, it will be much easier to push around, and vice versa. This is difficult to assess because it is an unknown factor.

If
\[ Military Advantage * Demand Value * Population Factor > Accessible System Value * Resistance \]
the NPR will accept the claim.

For example, if the NPR has militancy, determination and xenophobia all at 50, then the value to overcome for a Secondary system is 2.25. If there are no populations and you use ‘Demand Leave’ which is worth 2x, you will need a Military Advantage greater than 1.125. Making this demand will cause a negative relationship impact of \( 4^{2} * 3^{2} * (50/50) = 144 \). If you have a significant advantage in population in the system, then you require a smaller military advantage or can use a lesser demand.

If the NPR rejects your demand to withdraw, the protection status for that system for that NPR is reset to No Protection, so that further diplomatic penalties are not incurred. If you want to re-instate the demand (at whatever level), it will generate a new penalty.

If the NPR decides it must withdraw based on its assessment of the situation, it will evacuate its ships and transfer any colonies to your control. These will start at a status of Occupied. The system will be set to 'Alien Controlled' (Player controlled) from the perspective of the NPR and it will ignore the system when deploying forces. This will change if conflict breaks out.

Note that the player vs NPR and NPR vs player functionality for claiming systems are a little different. Both sides can send messages to each other and the types of messages are effectively the same. The difference is the method of delivery and the potential reaction. This is because I wanted to give the player maximum flexibility in Diplomacy, while still proving a structured approach for the NPR. For example, the player view of the NPR in terms of diplomatic points does not drop if the NPR ignores demands to leave. The player can decide whether it is necessary to go to war.

\section{NPR vs. NPR claims}\label{4_npr_vs_npr_claims}
Original post can be found
\href{http://aurora2.pentarch.org/index.php?topic=8495.msg118398#msg118398}{here}.
\newline\newline

NPR vs NPR Diplomacy works as a combination of NPR vs Players and Player vs NPR.

As described in section \ref{2_intrusion_into_npr}, when an NPR detects alien forces in a system that is claimed by the NPR, the NPR will issue a warning. When the target is a player this appears as an event message as per section \ref{2_intrusion_into_npr}. When the target is another NPR, the first NPR sets a protection status (in the same way as a player does in section \ref{3_claiming_npr_systems} that corresponds to the same demand level as it would send to a player.

For example: An NPR detects an alien force in a system that it claims and decides this represent a threat level of 12. If the alien is a player, the NPR will send a message to the player that will appear as an event. The message will be on the lines of "We demand you leave" and that message will continue to be sent each construction phase. If the target is another NPR (let's call this NPR-B), then NPR-A will set a protection status of 'Demand Leave' instead.

Next phase (or in some cases later in the same phase), NPR-B will see the withdrawal demand from NPR-A, just as it would see a similar demand from a player. It will react to that demand in exactly the same way except for one crucial difference; NPR-B will not reduce the diplomatic points for NPR-A.

So why all the messing about with slightly different methods for Player vs NPR, NPR vs Player and NPR vs NPR? Because NPRs, even though they are much smarter in C\#, will still not have the human capability to make intuitive estimates weighing the strategic benefit of claiming a system claim vs the potential downsides of reduced diplomatic relations. This strategic deficit in AI vs human ability is handled by the different reactions to claims.
\begin{itemize}
	\item Player vs NPR: The NPR will generally react negatively to being asked to leave a system, as that is a relatively easy to understand situation, and it can make a reasonable estimate of whether to abandon that system. The player does not react negatively to the NPR refusing to leave in game mechanics terms because the human player can make decisions himself about whether to treat the NPR differently. This also means that continual messages can be sent to remind the player without diplomatic penalties in-game.
	\item NPR vs Player: The NPR will react negatively to player forces being in one of its systems, as that is also a relatively easy to understand situation. The negative impact is based on the importance of the system and the size of the player force. The player does not react negatively to the NPR asking him to leave in game mechanics terms because the human player can make decisions himself about whether to leave or treat the NPR differently.
	\item NPR vs NPR: NPR-A will react negatively to NPR-B forces being in one of its systems, as that is also a relatively easy to understand situation. The negative impact is based on the importance of the system and the size of the NPR-B force. NPR-B will decide whether to leave the system but will not react negatively to being asked to do so. This allows the protection level to be reset each time without negative impact (so the NPR doesn't have to consider the huge variety of factors on when to make a new demand). Also, NPR-B may well regard the system as one of its own and will be making its own demand of NPR-A, in which case it will react negatively to a refusal from NPR-A.
\end{itemize}

The difference is that the NPR is always faced with an immediate decision and does not have to consider wider implications. The player has the ability to take those wider implications into consideration and is free to make his own decisions on relationships. When NPRs do confront each other, either one will leave because the system is not important or they will start making demands of each other, which takes care of the dual negativity. I know it sounds complex, but I think it the best option to handle the different situations.


\section{Restrictions on NPR claims}\label{5_restriction_on_npr_claimes}
Original post can be found
\href{http://aurora2.pentarch.org/index.php?topic=8495.msg118404#msg118404}{here}.
\newline\newline
There are several situations where NPRs will not make territorial claims:
\begin{itemize}
	\item If the NPR and the alien race share a capital system, no claims will be made in the capital system or in any adjacent system
	\item The NPR will not make claims against an alien race with whom it shares a Fixed Relationship due to a Truce Countdown
	\item The NPR will ignore claims from an alien race with whom it shares a Fixed Relationship due to a Truce Countdown and there will be no diplomatic penalty
	\item The NPR will not claim a system if there are alien populations with a total EM signature greater than \( 10\% * (Xenophobia / 100) \) of its own capital's EM signature and also greater than the total EM signatures of any AI populations in that system. The existence of populations will be based on intelligence data rather than current contacts.
\end{itemize}
The above is based on the concept that an AI is unlikely to claim a system where it knows there is a good chance that claim will cause a war. Note that from the NPR perspective an 'alien race' includes player races.


\section{Independence}\label{6_independence}
Original post can be found
\href{http://aurora2.pentarch.org/index.php?topic=8495.msg118467#msg118467}{here}.
\newline\newline
In C\#, you can declare a colony independent using a button on the Economics window. Colonies may also become independent in other situations, such as a rebellion following high unrest. Independence is far more complex than it first sounds, because the population will be under the control of a new race that is essentially a copy of the original race. The process is as follows:
\begin{itemize}
	\item The title of the new race will be based on the name of the newly-independent population.
	\item A new flag will be auto-selected and random naming themes chosen for classes, systems, etc.. Commander name themes will remain the same as the original race.
	\item The ranks of the new race will copy the ranks of the original race.
	\item Any ground forces at the population will be transferred to the new race.
	\item It is possible that an NPR population can become independent, in which cases it will retain the same tech but create a new design philosophy.
	\item The new race will start with an amount of wealth equal to total original race wealth * (independent pop size / total original race pop size before independence), which will be transferred from the original race.
	\item The new race will start with a number of commanders equal to original race number of \( commanders * (Independent Pop Size / Total Original) \) race pop size before independence). These are new commanders and not transferred from the original race.
	\item A top-level admin command will be created at the population.
\end{itemize}


The new race will gain the following knowledge from the original race:
\begin{itemize}
	\item The same galactic map, including map labels.
	\item All geological and gravitational survey data.
	\item All tech systems.
	\item How to build all ship components and missiles.
	\item All class designs.
	\item All ground unit class designs.
	\item All ground formation templates.
	\item All intelligence data, including alien races, classes, ships, sensors, weapons, populations and ground forces.
	\item A complete set of intelligence information on the original race which will be set up as a new alien race, with known systems, ships, etc.
	\item Control Race flags on galactic map.
	\item Protection Status settings for different combination of alien races and systems.
	\item Locations of ruins, anomalies, wrecks, etc..
	\item Event colours.
\end{itemize}

For manual independence, any naval forces will have to be transferred using the Transfer Fleet option. In the case of a rebellion, some ships may be transferred automatically.


\section{Banned bodies}\label{7_banned_bodies}
Original post can be found
\href{http://aurora2.pentarch.org/index.php?topic=8495.msg119432#msg119432}{here}.
\newline\newline

If a non-spoiler NPR has a relationship of neutral or higher with another race, it will generally avoid approaching 'banned bodies'.

An NPR will decide for itself which bodies are banned, but in general these will include:
\begin{enumerate}
	\item Bodies that have an alien race population of approximately ten million or more
	\item Bodies that are moons of any bodies in (1)
	\item Bodies that are moons and share the same parent body as any body in (1)
	\item Bodies on which the NPR already has a population will be exempt from the above rules
\end{enumerate}

NPRs will not create populations on banned bodies and will not attempt to conduct geological surveys on those bodies. The NPR will not generate points of interest within a few million kilometres of banned bodies. It is still possible that NPR ships will approach due to other considerations, such as moving between two points unrelated to banned bodies, but in general this should prevent the VB6 situation of NPR battle fleets making port visits to your home world.

The banned bodies list is updated at game launch and during each construction phase. Banned bodies do not exist for populations of races with which the NPR has a hostile relationship. If there are two populations on a planet, one of which is hostile to the NPR and one neutral, the body will not be banned.

For example, in the Space 1889 campaign, the Martians will generally avoid Venus, Earth, Luna, all the moons of Jupiter and all the moons of Saturn. They will still survey the Trojan asteroids and they still may pass close to the banned bodies when on an unrelated mission.

\textbf{NOTE:} \textit{I looked at various ways of applying this in reverse. The NPR would generate a list of important planets and check for player race ships within a certain range, perhaps ten million kilometres. If they were detected, that would trigger a response, even if the NPR would otherwise not object to the player being in the system. The problem is that the player would have to be checking each ship path to ensure that didn't happen. I even added code to avoid this problem by only flagging player ships that remained within ten million in two consecutive construction phases, but even that is not foolproof. Essentially, the player knows the NPRs is trying to avoid his populations and will react to NPR movements accordingly, but understanding that is much more difficult for the AI. In the end the game play benefit is outweighed by the considerable micro-management required on the part of the player, or by the amount of code that would be needed to avoid accidentally passing through restricted zones. In most situations, the player would want to avoid being detected anyway so this situation would usually only be relevant where a truce countdown is in effect and the player and the NPR share the same home system. The player can RP that situation if needed.}
\section{Diplomatic ships}\label{8_diplomatic_ships}
Original post can be found
\href{http://aurora2.pentarch.org/index.php?topic=8495.msg120024#msg120024}{here}.
\newline\newline

A Diplomatic ship is any ship equipped with a Diplomacy Module. These can be built by the player or by NPRs.

Diplomacy Modules and therefore Diplomatic Ships are important for communication attempts and essential for basic diplomacy (influencing an alien race to view your race more positively). See section \ref{1_basic_framework} for more details.

When a Diplomatic Ship is involved in diplomacy or communication attempts, the opposing race will know the origin of those messages. If the Diplomatic Ship is on opposing sensors, the identity of that ship will be noted in an event for the opposing race and its parent class will be flagged as a diplomatic vessel. If diplomacy is underway, the name of the Ambassador will also be passed to the opposing race.

If the Diplomatic Ship is not on opposing sensors, the location of the signal from that ship will be communicated to the opposing race. This may be a system body, a jump point or simply a point in space.

Any damage to NPR Diplomatic ships, regardless of whether the opposing race knows that status, will be treated as triple damage for the purposes for affecting diplomatic relations. If a diplomatic ship is attacked without an existing hostile relationship, the relationship will fall to -300 from the perspective of the owner of the ship (rather than the normal -100 for attacking when not hostile).


\newpage
\part{Star System Design}
Original post can be found

%\newline\newline
\section{Modifying stars}
Original post can be found

%\newline\newline
\section{Adding stars}
Original post can be found

%\newline\newline
\section{Modifying system bodies}
Original post can be found

%\newline\newline
\section{Deleting stars and system bodies}
Original post can be found

%\newline\newline
\section{Adding planets, comets and asteroid belts}
Original post can be found

%\newline\newline
\section{Adding moons and Lagrange points}
Original post can be found

%\newline\newline
\section{Deleting asteroids and Lagrange points}
Original post can be found

%\newline\newline

\newpage
\part{Weapons}
Original post can be found

%\newline\newline
\section{Missiles}
Original post can be found

%\newline\newline
\subsection{Missile updates}
Original post can be found

%\newline\newline
\subsection{Missile engines}
Original post can be found

%\newline\newline
\subsection{Missile launcher changes}
Original post can be found

%\newline\newline
\subsection{Box launcher reloading}
Original post can be found

%\newline\newline
\subsection{Missile thermal detection}
Original post can be found

%\newline\newline
\subsection{Magazine design}
Original post can be found

%\newline\newline

\section{Guns}
Original post can be found

%\newline\newline
\subsection{Meson update}
Original post can be found

%\newline\newline
\subsection{Turret update}
Original post can be found

%\newline\newline
\subsection{Beam weapon recharge}
Original post can be found

%\newline\newline
\subsection{Weapon failure}
Original post can be found

%\newline\newline
\subsection{Plasma carronades}
Original post can be found

%\newline\newline
\subsection{Particle lance}
Original post can be found

%\newline\newline

\section{Point defence}
Original post can be found

%\newline\newline
\section{Ordnance transfer mechanics}
Original post can be found

%\newline\newline
\section{Ordnance transfer orders}
Original post can be found

%\newline\newline
\section{Automated weapon assignment}
Original post can be found

%\newline\newline
\section{Atmoshpere and energy weapons}
Original post can be found

%\newline\newline
\section{Planetary bombardment}
Original post can be found

%\newline\newline

\newpage
\part{Ground Forces}
Original post can be found

%\newline\newline
\section{Unit design}
Original post can be found

%\newline\newline
\section{Formation templates}
Original post can be found

%\newline\newline
\end{document}