\documentclass[../../Aurora C# unofficial manual.tex]{subfiles}

\begin{document}
	\section{Ground-based geological survey}
	Original post can be found
	\href{http://aurora2.pentarch.org/index.php?topic=8495.msg107705#msg107705}{here}.
	\\\\
	
	Geological Survey Teams do not exist in C\# Aurora.
	
	Instead, a new ground unit component (100 tons) provides 0.1 survey points per day. Ground units with this component may be added to ground formations to provide a geological survey capability. All formations at the same population with a geological survey capability will combine their survey points to conduct a ground-based survey. This can only take place after the orbital survey is complete.
	
	Once the orbital survey of a system body is completed, the potential for a further ground survey will be revealed (None, Minimal, Low, Good, High, Excellent). The ground survey requires the same survey points as the orbital survey, except they are generated by ground forces. Only system bodies with a diameter of at least 4000 km will be eligible for a ground-based survey (in Sol that is Mercury, Venus, Earth, Mars, Ganymede, Callisto and Titan).

	Normal mineral generation (at system body creation) has three phases:
	\begin{enumerate}
		\item An overall roll for the potential for minerals to be present, based on radius, density and system abundance. If this roll fails, the body has no minerals.
		\item A roll for each type of mineral to be present, based on density and abundance. Duranium has twice the chance of any other mineral.
		\item A roll for the accessibility of each mineral generated in step 2). This is based on radius.
	\end{enumerate}

	Once the ground survey is completed (assuming potential is Minimal or higher), a new mineral generation roll will take place. For this roll:
	Step 1 is the same regardless of the potential.
	Step 2 is modified by the potential. Minimal is 25\% normal, Low is 33\% normal (same as teams in VB6), Good is 50\% normal, High is 100\% normal and Excellent is 200\% normal.
	Step 3 is modified by High (+ 0.1) and Excellent (+ 0.2). All others are same as normal.
	
	If a deposit of a mineral that didn't previously exist is generated by the ground survey, that deposit is added to the system body.
	If a mineral deposit is generated by the ground survey and a deposit of that mineral already exists on the system body, the existing deposit is changed to match the amount or accessibility (or both) of the ground survey deposit if the latter is greater.
	
	The chances that an eligible body (4000 km diameter) will have ground survey potential is equal to:
	None 60\%, Minimal 20\%, Low 10\%, Good 6\%, High 3\%, Excellent 1\%.
	
	For reference, in the Colonial Wars campaign, there are 2145 eligible bodies in 495 systems, so in general about 1.7 worlds per system will have potential of at least Minimal. About 1 system in 23 would have an Excellent potential world.
\end{document}