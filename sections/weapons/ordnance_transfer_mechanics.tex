\documentclass[../../Aurora C# unofficial manual.tex]{subfiles}

\begin{document}
	\subsection{Ordnance transfer mechanics}
	Original post can be found
	\href{http://aurora2.pentarch.org/index.php?topic=8495.msg104195#msg104195}{here}.
	\\\\
	
	In C\# Aurora, transferring ordnance is no longer instant and ships without specialised equipment cannot exchange ordnance in space. A ship can only receive ordnance at a Spaceport, an Ordnance Transfer Station, a ship with a Ordnance Transfer System, a base with a Ordnance Transfer Hub or in a military hangar bay.
	
	A new technology line - Ordnance Transfer Systems - provides the basis of the rate of ordnance transfer and allows ships to mount systems to transfer ordnance to or from other ships. The baseline system (Ordnance Transfer System: 40 MSP per Hour) sets the racial ordnance transfer rate at 40 MSP per hour and allows the use of the first ship-mounted Ordnance Transfer System. There are ten further steps in the tech progression with the highest tech system allowing ordnance transfer at 400 MSP per hour.
	
	Spaceports, Ordnance Transfer Stations or Ordnance Transfer Hubs will always use the highest tech ordnance transfer rate and can transfer ordnance to or from an unlimited number of ships simultaneously. However, the ships involved must be stationary. Hangar Bays also use the highest tech ordnance transfer rate (mainly to avoid multiple hangar bay types).
	
	Spaceports have increased in cost to 3600 BP but can now be moved by freighters. They are equal to four research facilities for transport purposes (or 80 factories). They retain their existing bonuses to loading and unloading cargo.
	
	Ordnance Transfer Stations are a new installation with a cost of 1200 BP. They do not require workers and can be moved by freighters. They have a transport size equal to 10 factories. Essentially, they are a cut-down version of a spaceport intended to facilitate ordnance transfer in forward areas, transferring ordnance between the surface of a planet and ships in orbit. They have no bonuses for loading or unloading cargo.
	
	An Ordnance Transfer Hub can be mounted on a ship. It is a commercial system with a research cost of 10,000 RP, build cost of 2400 BP and a size of 100,000 tons. In practical terms, this is likely to form part of a large, deep-space station, due to the size and cost, rather than being deployed on ammunition colliers that will accompany fleets.
	
	A Ordnance Transfer System is 500 tons and has a cost ranging from 20 BP to 200 BP, depending on the tech level. A ship with an Ordnance Transfer System can transfer ordnance to or from a single ship at once, so it will take some time to replenish a whole fleet, although this will improve with higher technology. At the early tech levels, the Ordnance Transfer System can only be used if both ships (collier and target ship) are both stationary. Underway Replenishment allows the transfer to take place while both ships are in the same fleet and underway. Priorities can be set for the ordnance transfer order when multiple ships are involved. The first Underway Replenishment tech allows ordnance transfer at 20\% of the normal rate (2500 RP), rising to 100\% with the highest tech (40,000 RP).
	
	Ordnance transfer order types will be adjusted to deal with the new requirements (which I will list in a separate post). Ordnance will be transferred during each movement increment as time passes until the target ship has full magazines.
\end{document}