\documentclass[../../Aurora C# unofficial manual.tex]{subfiles}

\begin{document}
	\subsection{Planetary bombardment}
	Original post can be found
	\href{http://aurora2.pentarch.org/index.php?topic=8495.msg107703#msg107703}{here}.
	\\\\
	
	In C\# Aurora, populations can be attacked by missiles and energy weapons. However, because missile warheads are area-effect weapons, they are much more effective at destroying the civilian population and any installations.
	
	Each installation type has a Target Size. The chance of each attack (either a missile or a single energy weapon) destroying an installation is equal to: Weapon Damage / Target Size. For example, a construction factory has a Target Size of 20, so a 10cm laser fired from orbit would have a 15\% chance to destroy the target (3 / 20). For the purposes of this check, missile warheads are treated as equal to 20x warhead strength. Therefore, a single 1 point warhead has a 100\% chance to destroy a construction factory.
	
	A single energy weapon can destroy only one target per hit. A missile warhead is applied until all damage is used. For example, a 5-point missile warhead is counted as 100. If the first installation hit is a construction factory, that factory is destroyed and the remaining damage reduced to 80. That damage is then applied the next installation hit and so on.
	
	Missile warheads cause radiation and dust levels to increase by an amount equal to their warhead size. Energy weapons increase the dust level by 5\% of their damage amount.
	
	Missile warheads inflict civilian casualties at the rate of 100,000 per point of damage. Energy weapons inflict civilian casualties at the rate of 2,000 per point of damage.
	
	Populations will no longer surrender purely due to orbital bombardment. You have to land ground formations to force a surrender.
	
	Energy weapons now provide a way to destroy the industry and infrastructure of a target population, without causing radiation or using up ordnance. However, this will require considerable effort for a large population and consume maintenance supplies due to weapon failures. It will also bring you within range of any ground-based energy weapons. Of course, it will usually be more beneficial to conquer the planet and gain the installations instead of destroying them.
	
	\begin{center}
		\begin{tabular}{|l|r|}
			\hline
			\textbf{Name} & \textbf{Target size} \\
			\hline
			Automated Mine & 20 \\
			\hline
			Cargo Shuttle Station & 200 \\
			\hline
			Civilian Mining Complex & 200 \\
			\hline
			Construction Factory & 20 \\
			\hline
			Conventional Industry & 20 \\
			\hline
			Deep Space Tracking Station & 40 \\
			\hline
			Fighter Factory & 20 \\
			\hline
			Financial Centre & 20 \\
			\hline
			Forced Labour Construction Camp & 80 \\
			\hline
			Forced Labour Mining Camp & 80 \\
			\hline
			Fuel Refinery & 20 \\
			\hline
			Genetic Modification Centre & 400 \\
			\hline
			Ground Force Training Facility & 400 \\
			\hline
			Infrastructure & 2 \\
			\hline
			Low Gravity Infrastructure & 2 \\
			\hline
			Maintenance Facility & 20 \\
			\hline
			Mass Driver & 20 \\
			\hline
			Military Academy & 400 \\
			\hline
			Mine & 20 \\
			\hline
			Naval Headqarters & 400 \\
			\hline
			Ordnance Factory & 20 \\
			\hline
			Ordnance Transfer Station & 200 \\
			\hline
			Refuelling Station & 200 \\
			\hline
			Research Lab & 400 \\
			\hline
			Sector Command & 400 \\
			\hline
			Spaceport & 1000 \\
			\hline
			Terraforming Installation & 100 \\
			\hline
		\end{tabular}
	\end{center}
\end{document}