\documentclass[../Aurora C# unofficial manual.tex]{subfiles}

\begin{document}
	\subsection{Missile updates}
	Original post can be found
	\href{http://aurora2.pentarch.org/index.php?topic=8495.msg103096#msg103096}{here}.
	\newline\newline
	
	The following changes will be made to missiles in C\# Aurora:
	\begin{enumerate}
		\item Missile Armour has been removed.
		\item Laser warheads have been removed (I may add these back at some point in the future).
		\item ECM is now a fixed 0.25 MSP for missiles. The 'Missile ECM' tech line has been removed and if a missile is equipped with ECM it will have the same ECM capability as the current racial ECM technology, The missile design will maintain that ECM capability and will not be upgraded if the racial tech improves. For each level of ECM, the missile will be 10\% harder to hit with energy weapons and will reduce the lock of missile fire controls by 10\%. This can be negated by linking a similar level of ECCM to the point defence fire controls.
		\item Missiles can be equipped with ECCM, which is a fixed 0.25 MSP. The missile ECCM level will be equal to the current racial ECCM tech. In C\# Aurora, the ECCM of missile fire controls will only affect the range at which the fire control can lock on. The ECCM of the missile itself will affect the chance of the missile striking its target, if that target has active ECM.
		\item Any missile sensor (active, thermal, EM or Geo) has to be a minimum of 0.25 MSP or it will have no effect.
		\item Missile series have been removed. Instead, there will be more detailed class loadout options.
	\end{enumerate}

	These changes will make electronic warfare much more important for missile combat. Missiles with ECM will become harder to shoot down and missiles without ECCM will have a reduced chance to hit targets equipped with ECM. Anti-missile missiles will either be less effective, or larger, vs ECM-protected missiles, while anti-ship missiles are likely to increase in size (and therefore reduce salvo sizes). Large volleys of size-1 missiles will be less effective in a heavy EW environment and no longer have a huge advantage in launching speed (due to the missile launcher changes).
	
	
	\subsection{Missile engines}
	Original post can be found
	\href{http://aurora2.pentarch.org/index.php?topic=8495.msg102804#msg102804}{here}.
	\newline\newline
	
	In C\#, Missile Engines follow the same size-based fuel consumption rules as Ship Engines using the formula:
	\[ Fuel Consumption = \sqrt{\frac{10}{Engine Size in HS}} \]
	
	The above increases the fuel consumption of missile engines based on size alone. However, VB6 also had a flat x5 multiplier for the overall fuel consumption for missile engines as they were treated as a different engine type than ship engines. As C\# is aiming for consistency between ship and missile engines, this x5 multiplier cannot remain as it was before. Removing the x5 multiplier entirely would cancel out the fuel consumption increase resulting from the changes in the size-based fuel consumption calculation. As one of the objectives of C\# is a reduction in missile ranges, a new rule is required that increases fuel consumption but that is still consistent with ship engines.
	
	Therefore, the calculation for fuel consumption based on boosting engines will now include an additional multiplier if the boost being used is higher than the maximum racial boost tech. Only missile engines have the capability to use higher boosts than the racial maximum, so this still allows consistency between ship and missile engines in the spectrum where they both operate. Once you move outside of the boost range possible for ships, additional fuel consumption can be added without breaking consistency. This rule adds a linear multiplier from 1x to 5x depending on the level of boost beyond the racial maximum. The formula is as follows:\newline
	
	if \( Boost Used > Max Boost Multiplier Tech \) then
	\[ High Boost Modifier = \frac{Boost Used - Max Boost Multiplier Tech}{Max Boost Multiplier Tech} * 4 + 1 \]
	
	So if a race has Max Boost Tech of 2x, any missile with a Boost Level of 2x or less will use the standard boost fuel modifier calculation of \( Boost Level^{2.5} \).
	
	Above a Boost Level of 2x, the linear High Boost Modifier will come into effect, reaching a maximum of 5x fuel consumption at 4x Boost Level.
	
	Here is a comparison between VB6 and C\# using MPD engines and an engine size of 1 MSP. The Max Boost Tech for this race is 2x:\newline\newline
	VB6 Missile Engine with 2x Boost\newline
	Engine Power: 1.6 \space\space\space Fuel Use Per Hour: 81.51 Litres\newline
	Fuel Consumption per Engine Power Hour: 50.944 Litres\newline
	Engine Size: 1 MSP \space\space\space Cost: 0.4\newline
	Thermal Signature: 1.6\newline
	Materials Required: 0.4x Gallicite\newline
	Development Cost for Project: 80RP\newline\newline
	C\# Missile Engine with 2x Boost
	Engine Power 1.60 \space\space\space Fuel Use Per Hour 76.8 Litres\newline
	Fuel Consumption per Engine Power Hour 48.0 Litres\newline
	Size 1.00 MSP (2.5 tons) \space\space\space Cost 0.80\newline
	Development Cost 80 RP\newline\newline
	Materials Required\newline
	Gallicite  0.80\newline\newline
	VB6 Missile Engine with 4x Boost\newline
	Engine Power: 3.2 \space\space\space Fuel Use Per Hour: 922.18 Litres\newline
	Fuel Consumption per Engine Power Hour: 288.182 Litres\newline
	Engine Size: 1 MSP \space\space\space Cost: 0.8\newline
	Thermal Signature: 3.2\newline
	Materials Required: 0.8x Gallicite\newline
	Development Cost for Project: 160RP\newline\newline
	C\# Missile Engine with 4x Boost\newline
	Engine Power 3.20 \space\space\space Fuel Use Per Hour 4344.5 Litres\newline
	Fuel Consumption per Engine Power Hour 1357.6 Litres\newline
	Size 1.00 MSP (2.5 tons) \space\space\space Cost 1.60\newline
	Development Cost 160 RP\newline\newline
	Materials Required\newline
	Gallicite  1.60
	
	
	\subsection{Missile launcher changes}\label{missile_launcher_changes}
	Original post can be found
	\href{http://aurora2.pentarch.org/index.php?topic=8495.msg102815#msg102815}{here}.
	\newline\newline
	
	Missile Launchers have undergone significant changes in C\# Aurora.
	\begin{enumerate}
		\item Fractional-size launchers can be created. The minimum is still 1 HS but a launcher can now be 1.1 HS, 2.7 HS, etc.
		\item The reduced-size launcher techs are all available immediately and do not need to be researched. This means box launchers are available from the start. The progression for reduced size launchers has been altered slightly:
		
		\begin{center}
			\begin{tabular}{|r|l|}
				\hline
				0.75 HS & 2x Reload \\
				\hline
				0.6 HS & 5x Reload \\
				\hline
				0.4 HS & 20x Reload \\
				\hline
				0.3 HS & 100x Reload \\
				\hline
				0.15 HS & 100x Reload (Box Launcher)\footnotemark \\
				\hline
			\end{tabular}
			\footnotetext[3]{Note that reload for this was x15 in VB6}
		\end{center}
	\end{enumerate}

	If a box launcher containing a missile is damaged, the missile will explode. The chance of this happening can be reduced by a new tech line. The first step reduces the explosion chance to 70\% for 1000 RP and the last step reduces to 5\% for 120,000 RP. In addition, Box launchers can only be reloaded in a hangar, or at an Ordnance Transfer Point (a Spaceport, Ordnance Transfer Station or Ordnance Transfer Hub). Reloading at an Ordnance Transfer Point is 10x slower than in a hangar (similar to the penalty for maintenance facilities in VB6 Aurora).
	
	The base reload rate for all missile launchers is now:
	\[ Missile Reload Rate = \frac{\sqrt{Missile Size} * 30 seconds * Reduced Size Modifier}{Reload Rate Tech} \]
	
	Assuming a race has reload rate tech of 3, a normal size 1 launcher will reload in 10 seconds, a size 4 will reload in 20 seconds and a size 9 will reload in 30 seconds. This change will dramatically reduce reload times for larger launchers.
	
	The change for box launcher reload rate from x15 to x100 is not as dramatic as it seems for larger missiles due to the new reduced reload times for larger missiles. However, it is still an significant increase from VB6. A size 4 missile mounted on a box launcher will now take about 1h 40m to reload in a hangar and about 17 hours for an ordnance transfer point. A size 6 is about 2 hours and 20 hours respectively.
	
	These changes are intended to:
	\begin{enumerate}
		\item Reduce the disadvantage of larger missiles
		\item Remove the realism issue of not having box launchers available at low tech yet make box launchers a more difficult decision vs standard-type launchers
	\end{enumerate}
	
	
	\subsection{Box launcher reloading}
	Original post can be found
	\href{http://aurora2.pentarch.org/index.php?topic=8495.msg109127#msg109127}{here}.
	\newline\newline
	
	In VB6 Aurora, box launchers can be reloaded in a hangar or at maintenance facilities. For C\# Aurora, box launchers can only be reloaded in a hangar, or at an Ordnance Transfer Point (a Spaceport, Ordnance Transfer Station or Ordnance Transfer Hub). Reloading at an Ordnance Transfer Point is 10x slower than in a hangar (similar to the penalty for maintenance facilities in VB6 Aurora).
	
	Because of the changes to maintenance facilities in C\# Aurora, it will be a lot easier to forward deploy facilities for full-size warships, both on planets and in space, which would increase the potential of box launchers if they could still use those facilities to reload, especially given they are immediately available in C\#. The introduction of ordnance-specific facilities for C\# provides a good alternative.
	
	The existing changes post for Missile Launchers section \ref{missile_launcher_changes} has been updated to take account of this new rule.
	
	
	\subsection{Missile thermal detection}
	Original post can be found
	\href{http://aurora2.pentarch.org/index.php?topic=8495.msg103478#msg103478}{here}.
	\newline\newline
	
	In VB6 Aurora, the thermal detection of missiles is based on the following formula:
	\[ Thermal Signature VB6 = \frac{Missile Size}{20} * \frac{Speed}{1000} \]
	
	I have no idea why I coded thermal detection for missiles to be based on size, although I am sure it seemed like a good idea at the time :). For C\# Aurora, missiles will use the same formula as ships for thermal signature:
	\[ Thermal Signature C\# = Max Engine Output * \frac{Current Speed}{Max Speed} * Thermal Reduction \]
	
	As missiles (for now anyway), don't have thermal reduction or an option to travel below maximum speed, their thermal signature is equal to the power of their engines. Combined with the changes to passive detection, this means that missiles in C\# Aurora will probably be detected by thermal sensors at much greater distances than in VB6 Aurora.
	
	
	\subsection{Magazine design}
	Original post can be found
	\href{http://aurora2.pentarch.org/index.php?topic=8495.msg107372#msg107372}{here}.
	\newline\newline
	
	There are several changes to magazine design for C\# Aurora.
	\begin{itemize}
		\item The 'ejection' tech line has been replaced by the Magazine Neutralisation System. It is functionally identical but in technobabble terms this is a system design to render missile warheads permanently inert in the event of damage to the magazine.
		\item Magazines have a base HTK number equal to the square root of their size (rounded down). in VB6 Aurora, all magazines have a base HTK of 1, regardless of size. It is still possible to add extra HTK in C\# by sacrificing internal space.
		\item The explosion chance for a magazine is divided by the square root of its size. For example, if a size 1 magazine has a base explosion chance of 15\%, the equivalent tech size 5 has an explosion chance of 6.71\%, the size 10 is 4.74\% and the size 20 is 3.35\%.
		\item If the ship has a Chief Engineer, any explosion chance (for magazines or engines) is reduced by his Engineering Bonus. So a 5\% explosion chance would be reduced to 3.5\% by a Chief Engineer with an Engineering bonus of 30\%.
		\item When a magazine is hit, a proportion of the remaining ordnance will be destroyed (based on destroyed magazine capacity / total ship magazine capacity). Any destroyed ordnance will explode with its full warhead strength.  In VB6, only ordnance beyond the remaining magazine capacity explodes and only at 20\% strength.
	\end{itemize}	
	In summary, magazine explosions in C\# Aurora will be much rarer, especially for larger ships, but far more devastating when they do occur.
\end{document}